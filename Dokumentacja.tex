\documentclass[10pt,a4paper]{report}
\usepackage[utf8]{inputenc}
\usepackage[T1]{fontenc}
\usepackage[MeX]{polski}
\usepackage{amsmath}
\usepackage{amsfonts}
\usepackage{amssymb}
\renewcommand{\thesection}{\arabic{section}}
\author{Kajetan Kaczmarek i Małgorzata Stawik}
\title{Dokumentacja WR}
\begin{document}
\begin{center}\begin{LARGE}
Dokumentacja Projektu WR \\
Robot Linerider\\
Małgorzata Stawik i Kajetan Kaczmarek
\end{LARGE}
\end{center}
\section{Wstęp}
W ramach projektu z przedmiotu "Wstęp do Robotyki" mieliśmy za zadanie wykonać robota śledzącego linie oraz przenoszącego ładunek. W ciągu 5 spotkań laboratoryjnych zbudowaliśmy fizyczną konstrukcję oraz stworzyliśmy oprogramowanie do jego obsługi
\section{Konstrukcja}
\begin{itemize}
\item
Pierwszy pomysł na konstrukcję robota zakładał użycie jednego silnika do napędu oraz drugiego sterującego osią skrętną.Po kilku próbach okazało się że robot miał duże problemy ze skręcaniem, a do tego wysoka konstrukcja negatywnie wpływała na stabilność.Przy drugiej próbie wykonaliśmy bardziej standardową konstrukcję z użyciem dwóch dużych silników do napędu dwóch kół oraz kulki jako trzeciego punktu podparcia z tyłu.Po jakimś czasie dodaliśmy także guzik do wyłączania programu dla ułatwienia testów.
\item
Jeśli chodzi o wykrywanie lini zamontowaliśmy na przodzie robota dwa czujniki światła które miały wykrywać kolor czarny i w takim wypadku skręcać w odpowiednim kierunku.Początkowo były one umieszczone zbyt wysoko, ale po opuszczeniu zaczęły zdawać egzamin.
\item
Do podnoszenia paczki zamontowaliśmy średni silnik z obrotowym hakiem do zaczepienia paczki.Poniżej niego znajduje się czujnik podczerwieni służący do wykrywania obiektu do podniesienia
\end{itemize}
\section{Oprogramowanie}
W zaprojektowanym oprogramowaniu użyliśmy szeregu funkcji:
\begin{itemize}
\item
 Funkcje pomocnicze countTheta i colorSum liczące pewne zmienne używane w algorytmach, tj. sumę wszystkich zmiennych RGB z danego czujnika światła oraz zmieniająca system na pseudoHSV
 \item
 checkRed i checkGreen - funkcje używane do sprawdzania w głównej pętli programu czy w danym momencie robot nie przejeżdża danym czytnikiem po kolorze zielonym lub czerwonym
 \item
 Funkcje dropBlock i pickUp, które służą do upuszczania i podnoszenia ładunku - składają się z małej pętli która wykonuje po kolei czynności wymagane w danej czynności
 \item
 Funkcja returnToLine która jest używana przy podnoszeniu i opuszczania klocka i służy do wrócenia na linię ruchu po wykonaniu zadania
 \item
 Funkcję steering i run realizujące główną funkcjonalność z użyciem algorytmu PID
\end{itemize}
\end{document}